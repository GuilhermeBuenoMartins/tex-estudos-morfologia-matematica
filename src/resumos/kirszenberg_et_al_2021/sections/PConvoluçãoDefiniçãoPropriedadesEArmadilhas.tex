Nesta seção, são detalhadas noções da p-convolução.

\subsection{Morfologia matemática em tons de cinza}
\label{subsec:morfologia-matematica-em-tons-de-cinza}

Em morfologia matemática, classicamente, uma imagem $f$ é representada como uma função 2D.
Defini-se $f: E \rightarrow \mathbb{R}$ com $x \in E$ sendo as coordenadas do píxel em um grid $E \subset \mathbb{Z}^{2}$ e os valores dos píxeis representados por $f(x) \in \mathbb{R}$.
Na morfologia matematática em tons de ciza, ambas a imagem $f$ e o elemento estruturante $b$ são valores reais (e não binário).
Ou seja, $b(x) \in \mathbb{R}$ e as operações de erosão $f \ominus b$ e dilatação $f \oplus b$ podem ser escrita como:

% Equation 1
\begin{equation}
(f \ominus b)(x)
    = \inf_{y \in E}\Big{\{} f(y) - b(x - y)\Big{\}}
    \label{eq:equation}
\end{equation}

% Equation 2
\begin{equation}
(f \oplus b)(x)
    = \sup_{y \in E}\Big{\{} f(y) + b(x - y)\Big{\}}
    \label{eq:equation2}
\end{equation}

O formualismo acima também é aplicável no uso de elementos estruturantes planos (binários) com

% Equation 3
\begin{equation}
    b(x) = \begin{cases}
               0        & \text{se } x \in B,       \\
               -\infty  & \text{caso contrário}     \\
    \end{cases}
    \label{eq:equation3}
\end{equation}
\\
onde $B \subset E$ é o suporte da função estruturante $b$.

\subsection{A média contra-harmônica e a p-convolução}
\label{subsec:a-media-contra-harmonica-e-a-p-convolucao}

A média ponderada de Lehmer, também conhecida como média contra-harmônica, é dada por um vetor não negativo $x = (x_{1}, x_{2}, \dots, x_{n}) \in (\mathbb{R}^{+})^{n}$, pesos não negativos $w = (w_{1}, w_{2}, \dots, w_{n}) \in (\mathbb{R}^{+})^{n}$ e com $p \in \mathbb{R}$.
Denotada por $CHM(x, w, p)$, é definida como:

% Equation 4
\begin{equation}
    CHM(x, w, p) = \frac{\sum_{i = 1}^{n} w_{i}x_{i}^{p}}{\sum_{i = 1}^{n} w_{i}x_{i}^{p-1}}
    \label{eq:equation4}
\end{equation}

Dentre os casos especiais, tem-se que $\lim_{p \to -\infty} CHM(x, w, p)$ é o mínimo dos elementos de x (ou seja, $\inf\{x_{i}\}$), bem como $\lim_{p \to +\infty} CHM(x, w, p)$ é o máximo dos elementos de x (ou, $\sup\{x_{i}\}$).

A p-convolução de uma imagem $f$ no píxel $x$ para valores do \emph{kernel} de convolução $w: W \subset E \to \mathbb{R}^{+}$ é definida como:

% Equation 5
\begin{equation}
    \begin{split}
        PConv(f, w, p)(x) = (f *_{p} w)(x) = \frac{(f^{p+1}*w)(x)}{(f^{p}*w)(x)} = \\
        = \frac{\sum_{y \in W(x)} f^{p+1}(y)w(x - y)}{\sum_{y \in W(x)} f^{p}(y)w(x - y)}
    \end{split}
    \label{eq:equation5}
\end{equation}
\\
onde $f^{p}(x)$ denota $(f(x))^{p}$ e, $W(x)$ é o suporte espacial do \emph{kernel} $w$ centrado em $x$.
Valendo-se das propriedades presentes na CHM, um comportamento morfológico é observado apartir de $p$.
Mais precisamente, uma pseudo-dilatação de $p \geq 0$ é alcançada quando $p \to +\infty$, bem como uma pseudo-erosão $p \leq 0$ para $p \to -\infty$.
Há, então, uma dominância do maior valor de píxel na vizinhança $W(x)$ para o píxel $x$ quando em uma dilatação (ou, em uma erosão, do menor valor de píxel na vizinha $W(x)$ para o píxel $x$) em uma soma ponderada.
Assintoticamente, $PConv(f, w, p)$ atua como uma diltação em tons de cinza não plana (ou como uma erosão em tons de cinza não plana) com função estruturante $b(x) = \frac{1}{p}\log(w(x))$:

% Equation 6
\begin{equation}
    \lim_{p \to +\infty} (f *_{p} w)(x) =
    \sup_{y \in W(x)} \Big{\{}f(y) + \frac{1}{p}\log(x(x - y))\Big{\}}
    \label{eq:equation6}
\end{equation}

% Equation 7
\begin{equation}
    \lim_{p \to -\infty} (f *_{p} w)(x) =
    \inf_{y \in W(x)} \Big{\{}f(y) - \frac{1}{p}\log(x(x - y))\Big{\}}
    \label{eq:equation7}
\end{equation}

De forma impírica, a aplicação das equações ~\ref{eq:equation6} e ~\ref{eq:equation7} são verdadeiras para $|p| > 10$.
Pode-se obter funções estruturantes planas usando pesos constantes nos \emph{kernels} de forma a satisfazer

% Equation 8
\begin{equation}
    w(x) = \begin{cases}
               1    & \text{, se } x \in W,                         \\
               0    & \text{, se } x \notin W \text{ e } |p| \geq 0 \\
    \end{cases}
    \label{eq:equation8}
\end{equation}
\\
e considerando, obviamente as limitações da definição de logaritmo.
Ou seja, quando $w(x) = 0$, temos as equações ~\ref{eq:equation6} e ~\ref{eq:equation7} como indefinidas.
Contudo, sendo a operação \emph{PConv} diferenciável, é compatível com a abordagem de aprendizagem via retropropagação com gradiente descendente.


\subsection{Limites da camada de p-convolução}
\label{subsec:limites-da-camada-de-p-convolucao}

Para que a \emph{PConv} seja definida em todas as suas entradas, são necessários que seus parâmetros $f$ e $w$ sejam estritamente positivos, além de outras limitações que devem ser evitadas.
São elas:

\begin{itemize}
    \item Se $w(x)$ ou $f^{p}(x)$ conter valores nulos, $\frac{1}{(f*_{p}w)(x)}$ não é definida;
    \item Se $f(x)$ contém valores nulos e $p$ é negativo, então $f^{p}(x)$ não é definida;
    e
    \item Se $f(x)$ conter valores negativos e $p$ diferir de nulo, tal que $p \notin \mathbb{R}$, $f^{p}(x)$ pode pertencer aos números complexos.
\end{itemize}

De forma geral, o cáculo a seguir é bastante utilizado para re-escalar valores de entrada no intervalo $[0, 1]$:

%Equation 9
\begin{equation}
    \text{maxmin}(f(x)) = \frac{f(x) - \min_{x \in E}f(x)}{\max_{x \in E}f(x) - \min_{x \in E}f(x)}.
    \label{eq:equation9}
\end{equation}

Sendo necessário re-escalar parâmetros de entrada entre camadas \emph{PConv} para se atingir uma pseudo-abertura ou pseudo-fechamento, por exemplo, aplica-se $1.0 + \text{maxmin}(f(x))$ de modo que a re-escala compreenda o intervalo $[1, 2]$.
Para se aplicar a re-escala, calcula-se:

% Equation 10
\begin{equation}
    f_{r}(x) = 1.0 + \text{maxmin}(f(x)) =
    1.0 + \frac{f(x) - \min_{x \in E}f(x)}{\max_{x \in E}f(x) - \min_{x \in E}f(x)}.
    \label{eq:equation10}
\end{equation}


