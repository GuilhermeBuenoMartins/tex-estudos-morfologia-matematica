Apresentadas as camadas morfológicas $\mathcal{L}$Morph e $\mathcal{S}$Morph inspiradas na camada \emph{PConv}, anteriormente baseadas nas propriedades do CHM visando atingir operações de dilatação e erosção em tons de cinza, foram aplicadas no conjunto de dados MNIST, 6 elementos estruturantes alvos de variados tamanhos e formas, para se avaliar o desempenho dessas camadas morfológicas de recuperar estes elementos estruturantes nas operações de dilatação, erosão, fechamento e abertura por meio treinamento.
No entanto, $\mathcal{S}$Morph baseado na função $\alpha$-softmax para os mesmos fins, sem as limitações encontradas em $\mathcal{L}$Morph e, consequentimente, \emph{PConv}, demonstrou resultados qualitativos e quantitativos melhores que os demais.

Os futuros trabalhos incluem investigação dos casos extremos das camadas bem como a integração destas a arquitetura de redes mais complexas e avaliações das mesmas em aplicações de processamento de imagens.