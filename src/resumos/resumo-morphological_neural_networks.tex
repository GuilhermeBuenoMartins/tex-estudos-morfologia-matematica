%! Author = gbueno
%! Date = 30/06/2022

% Preamble
\documentclass[12pt]{article}

% Packages
\usepackage{amsmath}
\usepackage{ragged2e}
\usepackage{hyperref}
\usepackage{amsfonts}

% Document
\begin{document}
    % Title
    \LARGE
    \centering
    Resumo de:\\ Davidson, J. L. and Hummer, F (1993). \textbf{Morphology Neural Networks – An Introduction with Applications}\\
    \vspace{0.25in}

    % Authory
    \large
    Guilherme Bueno Martins\\
    30 de junho de 2022\\
    \vspace{0.5in}
    \justifying


    \section{Introdução} \label{ch:introducao}\\
    Desenvolvida pela Força Aérea dos Estatos Unidos, a algebra de imagem é capaz de descrever qualquer transformação imagem-para-imagem que possa ser definida em processos de algoritmos finitos, como imagens em níveis de cinza.
    A algebra de imagem permite um produto generalizado entre uma imagem e um modelo, que pode ser usada para descrever uma computação de nó de rede neural.
    Dois especificos casos deste produto generalizado dão a computação neural clássica e a computação neural de morfologia.
    Como apresentado pelo artigo, a rede neural de morfologia provê um novo método de solução para o reconhecimento de padrões de tipo não linear, expresso pela Equação~\ref{eq:equation2}.\\
    A principal diferença entre a rede clássica e rede de morfologia é a computação algébrica de cada nó.
    Os resultados de cada nó antecedente são agrupados em $a = (a_1, a_2, \dots, a_n)$, multiplicados por seus pesos correspondentes $\{w_{ij}\}$, conforme a Equação~\ref{eq:equation1} e passados através da função de ativação do nó $f_i$:\\
    \begin{equation}
        b_j = f_i (\sum_{i=1}^{N} a_i w_{ji}).\\\label{eq:equation1}
    \end{equation}
    \begin{equation}
        b_j = f_i (\bigvee_{i=1}^{N} a_i + w_{ji}).\\\label{eq:equation2}
    \end{equation}
    A diferença da computação algébrica presente na rede neural de morfologia distingui-se pela estrutura algébrica subjacente dos valores utilizados.
    Usa-se em uma rede neural clássica números reais (\mathbb{R}) ou complexos (\mathbb{C}) cujos campos correspondentes podem ser respesentados como (\mathbb{R}, +, *, 0, 1) ou (\mathbb{C}, +, *, 0, 1).
    Por isso, 0 é a identidade para operação de adição (+), 1 é a identidade para a operação de múltiplicação.
    Já na rede neural de morfologia, sujetividade algébrica é a \texit{lattice} dos números reais extendidos,$\mathbb{R}_{\pm\infty} = \mathbb{R} \cup \{-\infty, +\infty\}$.
    A operação $\bigvee$, máximo ou limite superior mínimo, substitui a operação + acima, e a operção +, toma o lugar do operador *.
    O valor $-\inf$ cumpre o papel de 0, bem como valor de 0 substitui 1 no sistema número da rede de morfologia definida como:\\
    \begin{equation}
    (\mathbb{R}_{\pm\infty}, \bigvee, +, -\infty, 0)
        .\\\label{eq:equation3}
    \end{equation}
    É possível compreender a transformação a partir da substituição dos operadores lineares pelos seus operadores \textit{lattice} correspondentes.
    A transformação resultante é chamada de \textit{transformação lattice} e advém dos estudos da algebra matricial minimax.\\
    Na seção~\ref{sec:algebra-de-imagem-e-sua-relacao-com-redes-neurais}, detalha-se a importância que \textit{lattice} $\mathbb{R}_{\pm\infty}$ tem nas redes de morfologia.
    Na seção 3, as bases da teoria são dadas.
    Na seção 4, exemplos de redes neurais de morfologia são dados, com ou sem regras de aprendizagem, e para dois deles, prova-se teoremas de convergência.


    \section{Algebra de imagem e sua relação com redes neurais} \label{sec:algebra-de-imagem-e-sua-relacao-com-redes-neurais}\\
    Uma \textit{imagem} é uma função de um subconjunto $X$ de $N$ dimensões do espaço Euclidiano $\mathbb{R}^N$ para um conjunto de valores denotados por $\mathbb{F}$.
    Então, $a \in \mathbb{F}^X$ forma $\{(x,a(x)): x \in X, a(x) \in \mathbb{F}\}$, ou, se $X$ é finito com $q$ elementos, $\{(i,a(i)): i=1, \dots, q, a(i) \in \mathbb{F}\}$.
    Um \textit{template} é um elemento de $(\mathbb{F}^X)^Y$, onde $X \subset \mathbb{R}^N, Y \subset \mathbb{R}^M$.
    Equilalentemente, um \textit{template} $t$ é uma função, $t: Y \rightarrow \mathbb{F}^X$, formado por


    \section{A teoria das redes neurias de morfologia} \label{sec:a-teoria-das-redes-neurias-de-morfologia}


    \section{Exemplos de aplicações} \label{sec:exemplos-de-aplicacoes}
\end{document}